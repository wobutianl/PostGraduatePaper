\defaultfont
\BiAppendixChapter{\textbf{Abstract}}{}
{
In recent years, along with the in-depth development of geography,both in quantity and quality Geoscientific model resources has been greatly improved,As geography in mathematical language to describe the geological model,is the essence of geography,At the same time, along with the geography data access tools rich  and improvements .more convenient to get data .
However only through model transformation  geographical data can be effective into useful information,
Due to the Differences in the geographical data semantically and modeling environment,Caused a delay of application and reusable of geological model.
How to structure the bridge between geological model and multi-source geological information,increasing geoscientific models and upgrade the data convert rates of be an  important issues.

Classification of geo-data from multiple sources,is reunification of the geo-data management, query, retrieval and matching of critical methods,This article from the dual perspective of geological models and data,Built models of geography-based data classification, and data matching the specific areas of application of the model of data classification.
Data classification based on the construction of geological model based on parameterized.Design of XML-based geo-modeling method of semi-structured,Parameters of the geological model for the discovery of geoscientific data matching factor,Constructing geo-chain model of data manipulation and extraction.

For Applied Geoscience model and ultimate performance.
The main task of this paper is as follows:
1: analysis of the characteristics of multi-source geo-data,
Research on common data classification methods,And serve as a basis for model is proposed, and two layers of multi-source geological data classifications
While using non-relational database-MongoDB to achieve the unity of multi-source geological information management and organization.

2 analysis the needs of geoscience data model,Consider the geographic model reusability,
This article using parameterized model data and structured description of methods,
Achieve uniform expression of the geographic model data needs,
And parameter of model data and XML markup languages to non-formal and formal expression of the model,Improved model recognition and reusable.

3 develop a data manipulation and extraction process for model building,
Through the model parameters, get the model needs data formats and data types,
By matching the establishment of the database, and data manipulation library for data flow.

4 develop and implement a geographic model data-driven systems, The thought and design into three sections above,Form an integral whole,Finally proved the feasibility of multiple instances.\\
\\
}
\vspace{10cm}
\noindent{\textbf{Keywords:}} \quad{Geographic model, MongoDB, Multi-source data, metadata, XML}
